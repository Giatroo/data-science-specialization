%%%%%%%%%%%%%%%%%%%%%%%%%%%%%%%%%%%%%%%%%%%%%%%%%%%%%%%%%%%%%%%%%%%%%%%%%%%%%

\documentclass[12pt, a4paper, oneside]{article}

\input{/home/giatro/.config/user/giatro_packages.tex}

\input{/home/giatro/.config/user/giatro_macros.tex}

\title{Machine Learning Evalutation Metrics}
\date{\today}
\author{Lucas Paiolla Forastiere}

%%%%%%%%%%%%%%%%%%%%%%%%%%%%%%%%%%%%%%%%%%%%%%%%%%%%%%%%%%%%%%%%%%%%%%%%%%%%%
%%%%%%%%%%%%%%%%%%%%%%%%%%%%%%%%%%%%%%%%%%%%%%%%%%%%%%%%%%%%%%%%%%%%%%%%%%%%%

\begin{document}

\maketitle
\tableofcontents
\newpage

\section{Simple evaluation metrics}

In \textbf{classification}, the most common metric is the \textbf{accuracy}:

\[ E = \dfrac{\textrm{#Right predictions}}{\textrm{#Total predictions}}  \]

In \textbf{regression}, the most common metric is the \textbf{R^2 score}:

\[ r^2 = 1 - \dfrac{\Sigma(y_i-\hat{y}_i)^2}{\Sigma(y_i-\ol{y})^2} \]

However, there are many better evaluation metrics, and we're going to discuss
some of them here.

There could be many evaluation methods, but at the end, it depends on what you
want to do with the model you're creating.

Before proceding, let's understand why accuracy might not be the best metric.

One example is when we have \textbf{imbalanced classes}, which is when one of
the classes has lots more samples when the other classes. For instance, if we're
trying to predict if a credit card transaction is fraudulent, we may have many
more non fraudulent examples than fraudulent examples.

In that scenario, let's supose we got an accuracy of 99.9\%. That seens to be a
amazingly good classifier, but perhaps that 0.1\% os failness happens whenever
the transaction is fraudulent. We could have, for instance, a dummy model that
always predicts \textit{non-fraudulent} without even looking at the features!
That ridiculous model would have a very high accuracy, because most examples
indeed are \textit{non-fraudulent}, but it will fail every single that we have a
\textit{fraudulent} example, and that awful!

\section{Confusion Matrices}

When we have \textit{binary classifiers}, confusion matrices are very useful
because they allow us to visualize better how the model is making mistakes.
It consists in dividing the prediction classes and actual classes into a
$2\times 2$ matrix where the first row indicates when the model predicted the
positive class and the second the negative one, and the first column is all the
actual positive class and the second is the nevative one.

\begin{center}
\begin{tabular}{c c}
Predicted classes
\begin{tabular}{c|| m{3.65em} | m{3.6em} ||}
\multicolumn{3}{c}{Actual classes} \\
  & 1 & 0 \\ \hline\hline
1 & True positive & False positive \\ \hline
0 & False negative & True negative \\ \hline
\end{tabular}
\end{tabular}
\end{center}

By doing so, we divide the dataset into the cases when the model predicted
positive and that was right (\textit{true positive}); when the model predicted
negative and that was right (\textit{true negative}); when the model predicted
positive and that was wrong (\textit{false positive}); and when the model
predicted negative and that was wrong (\textit{false negative}).

Depending on our application, we might want to minimize false negatives and
don't care mush about false positives or vice-versa.

\subsection{Basic Evaluation Metrics}

We can also notice that the accuracy value is a summary of the table:

\[ E=\dfrac{TN+TP}{TN+TP+FN+FP} \]

Some other evaluation metrics we can define are \textbf{recall} and
\textbf{precision}:

\[ \textrm{Recall} = \dfrac{TP}{TP+FN} \]
\[ \textrm{Precision} = \dfrac{TP}{TP+FP} \]

The better the recall, the lower the number of \textit{false negatives} and
the better the precision, the lower the number of \textit{false positives}.

Therefore, we can use these values when avoiding whether \textit{false
negatives} or \textit{false positives}.

Recall is also known as \textit{True Positive Rate} (TPR) and there's also a
\textit{False Positive Rate} (FPR):

\[ FPR = \dfrac{FP}{TN+FP} \]

And we want that number to be the closest to zero as possible.

Having two (Recall and Precision) or even three (FPR) numbers is not ideal when
comparing two machine learning models. Sometimes, I'd better to have a simple
number such tal we could use to point at a model and say ``that's the better''.

The \textbf{F1-score} is a combination of precision and recall:

\[ F_1 = 2\cdot\dfrac{\textrm{Precision}\cdot
        \textrm{Recall}}{\textrm{Precision} + \textrm{Recall}} =
= \dfrac{2\cdot TP}{2\cdot TP+FN+FP} \]

This score is a specify case of a more general precision metric called the
\textbf{F-score}, which adds a parameter $\beta$:

\[ F_\beta = (1+\beta^2)\cdot\dfrac{\textrm{Precision}\cdot\textrm{Recall}}
{(\beta^2\cdot\textrm{Precision}) + \textrm{Recall}} =
\dfrac{(1+\beta^2)\cdot TP}{(1+\beta^2)\cdot TP + \beta\cdot FN + FP} \]

where we can set $\beta$ to emphasis recall or precision:
\begin{itemize}
  \item $0<\beta<1$ emphasizes precision;
  \item $1<\beta$ emphasizes recall;
\end{itemize}

%%%%%%%%%%%%%%%%%%%%%%%%%%%%%%%%%%%%%%%%%%%%%%%%%%%%%%%%%%%%%%%%%%%%%%%%%%%%%

\end{document}
